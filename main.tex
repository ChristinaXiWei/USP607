\documentclass{article}

% Language setting
% Replace `english' with e.g. `spanish' to change the document language
\usepackage[english]{babel}

% Set page size and margins
% Replace `letterpaper' with`a4paper' for UK/EU standard size
\usepackage[letterpaper,top=2cm,bottom=2cm,left=3cm,right=3cm,marginparwidth=1.75cm]{geometry}

% Useful packages
\usepackage{amsmath}
\usepackage{graphicx}
\usepackage[colorlinks=true, allcolors=blue]{hyperref}

\title{USP607: Community Development Seminar Study Plan}
\author{Christina Xi Wei}

\begin{document}
\maketitle

\section{Plan Description}

    This 10-week study plan is designed to guide USP607 Community Development Seminar. It aims to help student gains comprehensive knowledge about community development through its history and evolving process across the globe. The final delivery is a field paper which includes the topics discussed each week, and it should meet department's requirements on drafting the qualifying exam papers. The plan breaks into three sections: (1) history and theory of community development; (2)driving and resisting forces that change the dynamics of community; (3) case study of international community development. The detailed content and schedule for each section are listed below.

    To avoid derailing from the urban study discipline and the main goal of the plan, and at the same time address issues from both local and international dimensions, articles for each week are selected based on 1+1+1 principle, meaning each week includes (not limit to) one article from global perspective, one article from local U.S. perspective, and one article from urban studies perspective.

\section {Books}
\begin{enumerate}
    \item DeFilippis, James & Saegert, Susan. (2012). The community development reader(2nd edition). New York: Routledge.
    \item Roberts, J. T., & Hite, A. (Eds.). (2007). The globalization and development reader: Perspectives on development and global change. Malden, MA: Blackwell. 
    \item Campfens, H. (1997). Community development around the world : practice, theory, research, training.
\end{enumerate}


\section{Plan Schedule}

\subsection{History and Theory of Community Development}

\begin{itemize}
    \item Week1 Drafting study plan
    \item Week2 (10/4-10/10): Finalize plan + What is "Community"? What is "Community Development"?

        \begin{enumerate}
            \item[a] CDR, Part III: "Building and Organizing Community"
        \end{enumerate}

    \item Week3 (10/11-10/17): History and Theory of Community Development - An International Perspective

        \begin{enumerate}

            \item[a] Voth & Brewster (1989). An overview of international community development. In Christenson & Robinson (Eds.), Community development in perspective (1st ed., pp. 280-306). Iowa State Press
            \item[b] Campfens, H. (1997). International review of community development: Theory and Practice. In H. Campfens (ed.) Community development around the World: Practice, theory, research, training, (pp. 3-40) University of Toronto Press.
            \item[c] O’Conner, Alice. 2012. “Swimming Against the Tide: A Brief History of Federal Policy in Poor Communities.” In The Community Development Reader, edited by James DeFilippis and Susan Saegert. New York: Routledge, 11–29.
            \item[d] Woods, Clyde. 1998. “Regional Blocs, Regional Planning, and the Blues Epistemology in the Lower Mississippi Delta.” In Making the Invisible Visible: A Multicultural Planning History, edited by Leonie Sandercock. University of California Press, 78-99.

        \end{enumerate}
    \end{itemize}

\subsection{Driving and Resisting Forces}

\begin{itemize}
    \item Week4 (10/18-10/24): Economic factors
    \begin{enumerate}
            \item D&G (2007) Part I: Formative Approaches to Development and Social Change-Introduction (pp. 19-24)
                \begin{itemize}
                    \item 1. Manifesto of the Communist Party (1848); Alienated Labor (1844): Karl Marx & Friedrich Engels
                    \item 2. The Protestant Ethic and the Spirit of Capitalism (1905): Max Weber 
                    \item 3. The Stages of Economic Growth: A Non-Communist Manifesto: W.W. Rostow (1960) Roberts & Hite Part II: Dependency and Beyond: Introduction 
                    \item 5. The Development of Underdevelopment (1969): Andre Gunder Frank
                    \item 7. The Rise and Future Demise of the World Capitalist System: Concepts for Comparative Analysis (1979): Immanuel Wallerstein
                \end{itemize}
            \item D&G Part III: What is Globalization? Attempts to Understand Economic Globalization-Introduction (pp. 155-159)
                \begin{itemize}
                    \item 10. The New International Division of Labor in the World Economy (1980): Folker Fröbel, Jürgen Heinrichs, & Otto Kreye
                    \item 11. The Informational Mode of Development and the Restructuring of Capitalism (1989): Manuel Castells
                    \item 12. Cities in a World Economy (2000): Saskia Sassen 
                    \item 13. Globalization: Myths and Realities (1996): Philip McMichael
                \end{itemize}
        \end{enumerate}
    \item Week5 (10/25-9/31):Economic factors continue + Political factors
    \begin{enumerate}
            \item D&G Part IV: The Opportunities and Limits of Unfettered Globalization-Intro: Roberts & Hite (pp.259-262)
                \begin{itemize}
                    \item 16. In Defense of Global Capitalism (2003): Johan Norberg
                    \item 17. What Strategies are Viable for Developing Countries Today?: The World Trade Organization and the Shrinking of 'Development Space' (2003): Robert H. Wade
                    \item 18. Globalism's Discontents (2002): Joseph E. Stiglitz
                    \item 19. The New Global Economy and Developing Countries: Making Openness Work (1999) and Has Globalization Gone too Far? (1997): Dani Rodrik
                    \item 20. Industrial Convergence, Globalization, and the Persistence of the North-South Divide (1999): Giovanni Arrighi, Beverly J. Silver, and Benjamin Brewer
            \end{itemize}
            \item D&G (2007) Part V: Confronting Globalization-Introduction
                \begin{itemize}
                    \item 22. The Anti-Globalization Movement (2005): Jeffrey Sachs
                    \item 24. Environmental Advocacy Networks (1997): Margaret Keck and Kathryn Sikkink
                    \item 25. What Can We Expect from Global Labor Movements?: Five Commentaries (2002): Armbruster, Nash, Seidman, Ross, Appelbaum, Bickham-Mendez, & Bonacich
                    \item 26. Transnational Solidarity: Women's Agency, Structural Adjustment, and Globalization (2002): Manisha Desai
                    \item 27. Counter-Hegemonic Globalization: Transnational Social Movements in the Contemporary Global Political Economy (2005): Peter Evans
                \end{itemize}
    \end{enumerate}
    \item Week6 (11/01-11/07): Cultural factors
        \begin{itemize}
            \item  Gow, D. D. (2002). Anthropology and development: Evil twin or moral narrative? Human Organization, 61(4), 299-313.
            \item Fischer, E. F. (1996). Induced Culture Change as a Strategy for Socioeconomic Development: The Pan-Maya Movement in Guatemala. In E.F. Fischer & R.M. Brown (Eds.). Maya cultural activism in Guatemala U. Texas Press. (Chapter 3; for contents & intro: http://www.utexas.edu/utpress/excerpts/exfismay.html )
            \item Conklin, B. A. (2006). Environmentalism, global community, and the new indigenism. In M. H. Kirsch (Ed.), Inclusion and exclusion in the global arena. New York: Routledge.
            \item Shakow, M. N. (in press). Community as a genre of political practice: “organized civil society” and competing models of collectivity. In States of discontent: Political Dilemmas of the New Middle Classes in Bolivia (Chapter 6). Philadelphia: University of Pennsylvania Press.
            \item Mertz, E., & Timmer, A. (2010). Introduction-Getting it Done: Ethnographic Perspectives on NGOs. Political and Legal Anthropology Review, 33(2), 171–177.
            \item Deneulin, S., & Rakodi, C. (2011). Revisiting Religion: Development Studies Thirty Years On. World Development, 39(1), 45–54.
        \end{itemize}
    \item Week7 (11/08-11/14): Urbanization
      \begin{itemize}
            \item  Smith, N. (2002). New Globalism, New Urbanism: Gentrification as Global Urban Strategy. Antipode, 34(3), 427-450.
            \item Shatkin, G. (2004). Planning to Forget: Informal Settlements as 'Forgotten Places' in Globalising Metro Manila. Urban Studies, 41(12), 2469-2484.
            \item Pikholz, L. (1997). Managing Politics and Storytelling: Meeting the Challenge of Upgrading Informal Housing in South Africa. Habitat International, 21(4), 377-396.
            \item Kruger, J. S., & Chawla, L. (2002). “We know something someone doesn’t know”: Children speak out on local conditions in Johannesburg. Environment & Urbanization, 14(2), 85-96.]
            \item Dhiravisit, A. (2009). Government policy for urban poor community management in developing countries: Case study- Thailand. International Business and Economics Research Journal, 8(5), 89-97.
            \item Fang, Y. (2006). Residential Satisfaction, Moving Intention and Moving Behaviours: A Study of Redeveloped Neighbourhoods in Inner-City Beijing. Housing Studies, 21(5), 671 - 694.
            \item Smart, A., & Smart, J. (2003). Urbanization and the Global Perspective. An. Rev. of Anthropology, 32, 263-285.
        \end{itemize}
\end{itemize}

\subsection{Case study of International Community Development}

\begin{itemize}
    \item Week8 (11/15-11/21): Community development in North America and Europe
        \begin{enumerate}
            \item[ ] Campfens (1997) Part III "Canada" and Part IV "The Netherlands"
        \end{enumerate}
        
    \item Week9 (11/22-11/28): Community development in Asian, Latin America & Caribbean, Africa
        \begin{enumerate}
            \item[ ] Campfens (1997) Part VI "Ghana", Part VII "Bangladesh", and Part VIII "Chile"
        \end{enumerate}
    \item Week10 (11/29-12/05): Wrap-up and work on field paper
\end{itemize}


\section{Deliveries}

\subsection{Literature review} 
Each week, student will write a (proposed length 800~1000 words) literature review based on the topic and readings assigned for that week. These literature review will go into the field paper at the end of the term.

\subsection{Field paper}
The field paper should be constructed in a manner that fits the department's requirements on drafting a qualification exam paper. 
    
\section{Meeting Time}

TBD

\section{Reference syllabus}
    \begin{enumerate}
        \item "Community Development, Organizations, and Policies", taught by Doug Perkins at the Department of Human and Organizational Development in Vanderbilt University, Spring 2008
        \item "Global Dimensions of Community Development", taught by Doug Perkins at the Department of Human and Organizational Development in Vanderbilt University, Fall 2013
        \item "Foundations of Community Development", taught by Jennifer Tucker at School of Architecture and Planning in University of New Mexico, Spring 2021
    \end{enumerate}

\end{document}
